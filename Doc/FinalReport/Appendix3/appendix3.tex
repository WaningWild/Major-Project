\chapter{Code Examples}

\section{Interest Point struct}

The interest point data structure. Made to hold any and all information needed for each interest point.

\begin{verbatim}
 struct interestPoint {
   cv::Point2f currentFeature;
   cv::Point2f previousFeature;
   double MotionVectorAngle;
   bool isForeground;
   int clusterLabel;
 };
\end{verbatim}

\section{Clustering}
This is the clustering function I used for clustering my points.

\begin{verbatim}
/****************************************************************
 ****************************************************************
 ** Function: clusterPoints()                                   *
 ** Purpose : This function will perform the basics to start    *
 **             clustering the points.
 ** Params  : 							                        *
 ** Returns : 							                     	*
 ****************************************************************
*/
std::vector<cluster> clusterPoints(std::vector<interestPoint> *interestPoints, int splitLevel) {

  ///Sort the features in descending order based on number of occurences of corresponding motion vector angle
  mergeSort(*interestPoints);

  std::vector<cluster> allClusters;
  cluster current_cluster;
  current_cluster.centroid = interestPoints->at(0);
  current_cluster.label = 0;

  for (size_t i = 0; i < interestPoints->size(); i++) {
    if (std::abs(interestPoints->at(i).MotionVectorAngle - current_cluster.centroid.MotionVectorAngle) < angleThreshold) {
      current_cluster.interestPoints.push_back(interestPoints->at(i));
    }
    else {
      allClusters.push_back(current_cluster);
      current_cluster.centroid = interestPoints->at(i);
      current_cluster.interestPoints.clear();
      current_cluster.interestPoints.push_back(interestPoints->at(i));
      current_cluster.label = current_cluster.label + 1;
    }
  }

  return allClusters;

}
\end{verbatim}