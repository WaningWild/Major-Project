\chapter{Evaluation}

%Examiners expect to find in your dissertation a section addressing such questions as:

%\begin{itemize}
%   \item Were the requirements correctly identified? 
%   \item Were the design decisions correct?
%   \item Could a more suitable set of tools have been chosen?
%   \item How well did the software meet the needs of those who were expecting to use it?
%   \item How well were any other project aims achieved?
%   \item If you were starting again, what would you do differently?
%\end{itemize}

%Such material is regarded as an important part of the dissertation; it should demonstrate that you are capable not only of carrying out a piece of work but also of thinking critically about how you did it and how you might have done it better. This is seen as an important part of an honours degree. 

%There will be good things and room for improvement with any project. As you write this section, identify and discuss the parts of the work that went well and also consider ways in which the work could be improved. 

%Review the discussion on the Evaluation section from the lectures. A recording is available on Blackboard. 
\section{Process}
I believe my process/method was well thought out, it incrementally improved through out the project and came out well adapted to a project were there is little experience with the technology in use. This process however has drawbacks. If the process were to be applied to a different project in it's current state, where there a very few sprint backlog stories and a small WIP limit, it would likely slow down development if the technology was well known by the person undertaking the project. I'd consider it highly likely that if this project were to continue forwards, this process would adapt to my knowledge of the technology and eventually the limits on backlog size and WIP would increase again.

While I believe the process itself is solid, I do think the implementation of the process could have been better. For example, the kanban board was a physical board located only in one workspace. This became an issue when I moved through different workspaces and then found either that I didn't know what stories I was supposed to be working on or that if I did, I couldn't rightly then move on because I didn't know what was in the backlog. This is an inherent problem in having a physical information radiator but it could be solved through the use of manual syncing between a physical and purely digital board. In my case, what I should have done is implement a digital board as well as the physical board and thus helping improve work flow outside of the single workspace, and therefore increasing output.

Another issue with this process was that it relies on a single person reviewing sprints and progress, which makes it very easy to miss potential issues.It also makes it easier to write off what could be an error in the process as personal laziness or write off personal laziness as an error in the process. I personally can't think of a good solution to this in a single person project, with my best solution being to bring another person in to assess the work completed. This solution however doesn't work so well when this other person has no knowledge of the system being worked on and any other solution requires extreme diligence in terms of logging what you're doing exactly when you do it. I feel this diligent logging procedure would detract from the work at hand, as well as appearing to be over bearing for smaller projects.

\section{Design decisions}
This project was a project of simple design but complex algorithmic difficulty. I feel like this project lended itself to a functional paradigm, following a strict flow from step to step and avoiding the use of objects. I feel like an object oriented design would have been overkill for a project like this, as there would be very few objects interacting with each other. 

My design could also be viewed as overly simplistic and while I don't believe the design is overly simplistic, more upfront design would have improved the overall structure and readability of this project, as well as preventing some issues such as refactoring in the interest point data structure and refactoring out the tiled code.


\section{End result}
Overall, I think my project scope was larger than expected at the start. This, coupled with my inexperience with the technologies in use, contributed heavily to the final state of this project being unfinished. The project also presented a difficult to solve problem in and of itself, with multiple research projects having been dedicated to monocular flying obstacle avoidance in the past.

In retrospect, this large project scope could have been avoided by spending more time researching to find a more relevant paper to base the project off of, as adapting a paper designed for a different system proved to introduce a lot of extra intricacies that were not immediately obvious. It also would have been beneficial for me to have looked more in-depth at the papers at the beginning of the process, as that would have avoided issues like lack of libraries and confusion at particular parts of the paper(namely, the clustering). 