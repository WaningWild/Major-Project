\documentclass[11pt,fleqn,twoside]{article}
\usepackage{makeidx}
\makeindex
\usepackage{palatino} %or {times} etc
\usepackage{plain} %bibliography style 
\usepackage{amsmath} %math fonts - just in case
\usepackage{amsfonts} %math fonts
\usepackage{amssymb} %math fonts
\usepackage{lastpage} %for footer page numbers
\usepackage{fancyhdr} %header and footer package
\usepackage{mmpv2} 
\usepackage{url}

% the following packages are used for citations - You only need to include one. 
%
% Use the cite package if you are using the numeric style (e.g. IEEEannot). 
% Use the natbib package if you are using the author-date style (e.g. authordate2annot). 
% Only use one of these and comment out the other one. 
\usepackage{cite}
%\usepackage{natbib}

\begin{document}

\name{Joseph James}
\userid{jgj2}
\projecttitle{Mobile Vision Based Risk Detection}
\projecttitlememoir{Mobile Vision Based Risk Detection} %same as the project title or abridged version for page header
\reporttitle{Outline Project Specification}
\version{0.1}
\docstatus{Draft}
\modulecode{CS39440}
\degreeschemecode{G400}
\degreeschemename{Computer Science}
\supervisor{Myra Wilson} % e.g. Neil Taylor
\supervisorid{mxw}
\wordcount{}

%optional - comment out next line to use current date for the document
%\documentdate{10th February 2014} ma
\mmp
 
\setcounter{tocdepth}{3} %set required number of level in table of contents



%==============================================================================
\section{Project description}
%==============================================================================
This project will be focused on developing a risk detection system for a parrot AR Drone 2.0 using a vision-based system. This will allow the drone to focus avoid potential upcoming risks through it's main camera, as the drone doesn't have any other methods for avoiding collisions due to it normally being manually controlled.\\

This would also provide a myriad of possibilities in what the drone could help do; this includes things like assistance for those poor of sight, leading in search and rescue through sight and general collision avoidance through sight. Currently, the drone operates via user input for control and relies on their vision through the cameras on the drone and their outside view of the drone to avoid collisions. This is fine for personal use and some photographic use, however for autonomous flight, the drone needs to be able to detect these collisions on it's own.\\

In order to do this, a computer vision based system to attempt to detect particular risks through the cameras would be implemented. It's likely that the principal of relative motion in optical flow will be used to do this, as objects that are closer to the drone will appear to be moving faster, however this will require further investigation into different methods of determining risks through vision. The plan will also involve starting with a very limited scope of risks, for example I would use only red shapes initially and then try to generalize from there. This allows for an increase in scope as I move forward in the project, while limiting it to make sure the scope is suitable at the beginning. \\

In order to complete this project, I will be following an agile methodology consisting of sprints, iterative process improvement and a personal kanban system\cite{PersonalKanban}. Early on in my process, I will be setting up my system for my workflow and writing up the methodology and my motivation behind my decisions.
%==================================================================== ==========
\section{Proposed tasks}
%==============================================================================
\begin{description}
  \item[Set up and write up methodology]
    This task will allow me to have a clear idea of how I'm going to progress with further work, keeping me on task and always making progress. It also helps me keep an idea of how my methodology has changed, as I'll have a written record of how it started. This will be done very early on in the project, before any work commences.
  \item[Set up version control]
    This version control will be hosted on github and used for back-up and easy roll-backs. This is necessary in order to make sure any mistakes in development can be rectified easily and it also allows for tracking of the progress of the project overall.
  \item[Research]
  This task is to gain further understanding of the technology I'll be using and the concepts I'll be basing my system off.
    \begin{description}
      \item[Vision based risk detection systems]
        Here, research on preexisting vision based systems for risk or collision detection will take place. This is done to gain a clear view of the type of system that should be implemented, what different layers it will need and anything else it will entail.
      \item[Parrot AR Drone 2.0 API/SDK]
        In this task, there will be investigation of the API and SDK of the drone I will be using as a platform for my system. This is to gain an understanding of the system before development gets underway, to improve development time. 
    \end{description}
\end{description}

%==============================================================================
\section{Project deliverables}
%==============================================================================

%
% Start to comment out / remove the following lines. They are only provided for instruction for this example template.  You don't need the following section title, because it will be added as part of the bibliography section. 
%
%==============================================================================
%\section*{Your Bibliography - REMOVE this title and text for final version}
%==============================================================================
%
%You need to include an annotated bibliography. This should list all relevant web pages, books, journals etc. that you have consulted in researching your project. Each reference should include an annotation. 
%
%The purpose of the section is to understand what sources you are looking at.  A correctly formatted list of items and annotations is sufficient. You might go further and make use of bibliographic tools, e.g. BibTeX in a LaTeX document, could be used to provide citations, for example \cite{NumericalRecipes} \cite{MarksPaper} \cite[99-101]{FailBlog} \cite{kittenpic_ref}.  The bibliographic tools are not a requirement, but you are welcome to use them.   
%
%You can remove the above {\em Your Bibliography} section heading because it will be added in by the renewcommand which is part of the bibliography. The correct annotated bibliography information is provided below. 
%
% End of comment out / remove the lines. They are only provided for instruction for this example template. 



\nocite{*} % include everything from the bibliography, irrespective of whether it has been referenced.

% the following line is included so thadded to the previous page for the bibliography. To address this, a newline is added so that it appears on the first page for the bibliography. 
at the bibliography is also shown in the table of contents. There is the possibility that this is \newpage
\addcontentsline{toc}{section}{Initial Annotated Bibliography} 

%
% example of including an annotated bibliography. The current style is an author date one. If you want to change, comment out the line and uncomment the subsequent line. You should also modify the packages included at the top (see the notes earlier in the file) and then trash your aux files and re-run. 
%\bibliographystyle{authordate2annot}
\bibliographystyle{IEEEannot}
\renewcommand{\refname}{Annotated Bibliography}  % if you put text into the final {} on this line, you will get an extra title, e.g. References. This isn't necessary for the outline project specification. 
\bibliography{mmp.bib} % References file

\end{document}
