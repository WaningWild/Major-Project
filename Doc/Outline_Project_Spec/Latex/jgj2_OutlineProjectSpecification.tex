\documentclass[11pt,fleqn,twoside]{article}
\usepackage{makeidx}
\makeindex
\usepackage{palatino} %or {times} etc
\usepackage{plain} %bibliography style 
\usepackage{amsmath} %math fonts - just in case
\usepackage{amsfonts} %math fonts
\usepackage{amssymb} %math fonts
\usepackage{lastpage} %for footer page numbers
\usepackage{fancyhdr} %header and footer package
\usepackage{mmpv2} 
\usepackage{url}

% the following packages are used for citations - You only need to include one. 
%
% Use the cite package if you are using the numeric style (e.g. IEEEannot). 
% Use the natbib package if you are using the author-date style (e.g. authordate2annot). 
% Only use one of these and comment out the other one. 
\usepackage{cite}
%\usepackage{natbib}

\begin{document}

\name{Joseph James}
\userid{jgj2}
\projecttitle{Mobile Vision Based Risk Detection: \\Using computer vision to detect potential risks in a Parrot AR.Drone 2.0}
\projecttitlememoir{Mobile Vision Based Risk Detection} %same as the project title or abridged version for page header
\reporttitle{Outline Project Specification}
\version{1.0}
\docstatus{Release}
\modulecode{CS39440}
\degreeschemecode{G400}
\degreeschemename{Computer Science}
\supervisor{Myra Wilson} % e.g. Neil Taylor
\supervisorid{mxw}
\wordcount{}

%optional - comment out next line to use current date for the document
%\documentdate{10th February 2014} ma
\mmp
 .
\setcounter{tocdepth}{3} %set required number of level in table of contents



%==============================================================================
\section{Project description}
%==============================================================================
This project will be focused on developing a risk detection system for a parrot AR Drone 2.0 using a vision-based system. This will allow the drone to focus avoid potential upcoming risks through it's main camera, as the drone doesn't have any other methods for avoiding collisions due to it normally being manually controlled.\\

This would also provide a myriad of possibilities in what the drone could help do; this includes things like assistance for those poor of sight, leading in search and rescue through sight and general collision avoidance through sight. Currently, the drone operates via user input for control and relies on their vision through the cameras on the drone and their outside view of the drone to avoid collisions. This is fine for personal use and some photographic use, however for autonomous flight, the drone needs to be able to detect these collisions on it's own.\\

In order to do this, a computer vision based system to attempt to detect particular risks through the cameras would be implemented. It's likely that the principal of relative motion in optical flow will be used to do this, as objects that are closer to the drone will appear to be moving faster, however this will require further investigation into different methods of determining risks through vision. The plan will also involve starting with a very limited scope of risks, for example detecting only red shapes initially and then generalizing from there. This limits the scope at the start of the project to a suitable level, then allows for further expansion of the scope as the project goes on.\\

In order to complete this project, an agile methodology consisting of sprints, iterative process improvement and a personal kanban system\cite{PersonalKanban} will be used. Early on in this process, a system for my workflow will be set up and the methodology and my motivation behind my decisions will be written up.
%==================================================================== ==========
\section{Proposed tasks}
%==============================================================================
\begin{description}
  \item[Set up and write up methodology]
    This task will allow for a clearer idea of how the work will progress, keeping work moving forward throughout. It also helps keep an idea of how the methodology has changed, as there will be a written record of how it started. This will be done very early on in the project, before any work commences, at the same time as the research.
  \item[Set up version control]
    This version control will be hosted on github and used for back-up and easy roll-backs. This is necessary in order to make sure any mistakes in development can be rectified easily and it also allows for tracking of the progress of the project overall. Seeing as this is a very easy task, it's to be done within the first week.
  \item[Research]
  This task is to gain further understanding of the technology which will be used and the concepts that the system will be based off of. It will be done within the first 2 or 3 weeks of the project.
    \begin{description}
      \item[Vision based risk detection systems]
        Here, research on preexisting vision based systems for risk or collision detection will take place. This is done to gain a clear view of the type of system that should be implemented, what different layers it will need and anything else it will entail.
      \item[Parrot AR Drone 2.0 API/SDK]
        In this task, there will be investigation of the API and SDK\cite{DroneAPI} of the drone being used as a platform for the system. This is to gain an understanding of the system before development gets underway, to improve development time. 
    \end{description}
  \item[Development]
    This will be split into a few sub tasks and will be started once research is completed:
    \begin{description}
      \item[Vision]
        This will be the development of the Vision-based system and is likely to be the main part of the project. This will involve detection of potential risks, determining if they are a risk and sending the appropriate information to the controller.
      \item[Drone controller]
        This is the part of the project that will be controlling the drone itself, sending commands to perform particular movements. This is necessary for applying the vision system to the drone, as it controls the drones movements and sends the camera feeds into the vision system.
      \item[Testing]
        This task is intertwined in both the previous development tasks and will be done in both. During both tasks, I will be doing continued testing of the vision and controller, in order to make sure it's working correctly and does what it is intended to do. Also, integration testing will be taking place to check overall functionality when applied to the drone. 
    \end{description}
  \item[Meetings and progress/process tracking]
    Throughout the project there will be weekly supervisor meetings to discuss progress and to help solve any problems that can't otherwise be solved. In order to keep these meetings properly informed, progress and process tracking will be taking place through the form of documentation in the repository. This will include progress made, any changes that have been made to the process and reasoning behind the decisions I made.
  \item[Demonstration preparation]
    During the time on this project, there will be 2 demonstrations, both of which will need to be prepared for beforehand. This preparation will involve implementation on the drone, making sure all the technology is usable (charged, working, accessible) and preparation of what information needs to be expressed to show progress. For the mid-project demonstration it's hoped that basic risk detection of a simplified risk (E.g A designed risk like a red square/circle). In the final demonstration, the drone should have a fully implemented vision system, with appropriate reactions to the risks. These plans are subject to change however, depending on how much has been completed at the time of the presentations.
  \item[Final report write up]
    The final report will mostly be written once development has finished, however the report will have some parts that are written during development, after particular parts of development have finished. This document will have sections that can't be completed during development however, such as critical analysis of the process over the project and progress made. 
\end{description}

%==============================================================================
\section{Project deliverables}
%==============================================================================
\begin{description}
  
  \item[Drone software]
    Any software produced as part of this project is contained in this deliverable. This will be submitted in the technical hand-in and will also be available within version control. This will also contain any extra material used as part of the technical work such as libraries or any other acknowledged work from another source.
  \item[Demonstration notes]
    These will be notes used for the demonstrations, consisting of two sets of notes. One of the sets will be for the mid-project demonstration and the other for the final demonstration. They are to be added to the appendix of the final report.
  \item[Research notes]
    As part of the research into different vision techniques to be used in this project, a set of notes on these techniques will be produced. These will be discussing the advantages and disadvantages of different techniques, along with critical analysis for the purpose of this project. These will be included in the appendix of the final report.
  \item[Final report]
    The final report will discuss the work done on the project and acknowledge any work used for the project. It will also include an appendices, which will document additional supporting material. 
\end{description}
%
% Start to comment out / remove the following lines. They are only provided for instruction for this example template.  You don't need the following section title, because it will be added as part of the bibliography section. 
%
%==============================================================================
%\section*{Your Bibliography - REMOVE this title and text for final version}
%==============================================================================
%
%You need to include an annotated bibliography. This should list all relevant web pages, books, journals etc. that you have consulted in researching your project. Each reference should include an annotation. 
%
%The purpose of the section is to undeirstand what sources you are looking at.  A correctly formatted list of items and annotations is sufficient. You might go further and make use of bibliographic tools, e.g. BibTeX in a LaTeX document, could be used to provide citations, for example \cite{NumericalRecipes} \cite{MarksPaper} \cite[99-101]{FailBlog} \cite{kittenpic_ref}.  The bibliographic tools are not a requirement, but you are welcome to use them.   
%
%You can remove the above {\em Your Bibliography} section heading because it will be added in by the renewcommand which is part of the bibliography. The correct annotated bibliography information is provided below. 
%
% End of comment out / remove the lines. They are only provided for instruction for this example template. 


\nocite{*} % include everything from the bibliography, irrespective of whether it has been referenced.

% the following line is included so that the bibliography is also shown in the table of contents. There is the possibility that this is added to the previous page for the bibliography. To address this, a newline is added so that it appears on the first page for the bibliography. 
%\newpage
\addcontentsline{toc}{section}{Initial Annotated Bibliography} 

%
% example of including an annotated bibliography. The current style is an author date one. If you want to change, comment out the line and uncomment the subsequent line. You should also modify the packages included at the top (see the notes earlier in the file) and then trash your aux files and re-run. 
%\bibliographystyle{authordate2annot}
\bibliographystyle{IEEEannot}
\renewcommand{\refname}{Annotated Bibliography}  % if you put text into the final {} on this line, you will get an extra title, e.g. References. This isn't necessary for the outline project specification. 
\bibliography{mmp.bib} % References file

\end{document}
